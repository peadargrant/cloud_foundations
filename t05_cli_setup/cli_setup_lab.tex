\documentclass{pgnotes}

\title{Command Line Interface Setup}

\begin{document}

\maketitle

This lab will setup your local PC environment and familiarise you with key operations we will use each week.
The lab PCs have most of the software required already installed.

\section{Using your own computer}

If you're on a lab computer then skip ahead to \autoref{sec:powershell-execution-policy}.
I only support the lab configuration but you are welcome to try out and use your own machine (incl. MacOS, Linux etc) if you can support it yourself.
All of the software used here is cross-platform.

Mac, Linux, Unix users will have no problems installing the AWS CLI, and
the commands work identically to those on Windows.


\subsection{Software installation}

Required software:
\begin{description}
\item[PowerShell] on non-MS systems\\from \url{https://github.com/PowerShell/PowerShell/releases/tag/v7.2.6}
\item[Git] from \url{https://git-scm.com/downloads}
\item[AWS CLI] from \url{https://aws.amazon.com/cli/}
\end{description}

\subsection{PowerShell usage}

The difference between Mac/Linux and Windows centres on the use of
Bash/zsh by Mac/Linux/UNIX vs PowerShell on Windows. The AWS CLI is
perfectly scriptable using Bash, particularly in conjunction with
\texttt{jq} to parse JSON. However, some of the scripts you will be
provided with in this module will be PowerShell only due to time
constraints.

The good news is that PowerShell Core 7 can be installed easily on a Mac
with no issues. You \emph{do not} need a Windows VM on your Mac to use
any of the PowerShell or AWS commands in \emph{this} course. Please go
to the \href{https://github.com/PowerShell/PowerShell}{PowerShell page
on GitHub} for instructions.

When you have PowerShell installed, open the Terminal app and type
\texttt{pwsh} and you'll be at a PowerShell prompt.

\section{PowerShell execution policy}
\label{sec:powershell-execution-policy}

(Not needed on non-Windows systems.)
We will use a number of PowerShell scripts throughout the module. 
PowerShell by default will not allow scripts to run that were downloaded from online.
Run the following command to change this behaviour:

\begin{verbatim}
Set-ExecutionPolicy -ExecutionPolicy RemoteSigned -Scope CurrentUser
\end{verbatim}

Answer \textit{all} when prompted.


\section{Setup AWS command-line interface}
\label{sec:setup-aws-command-line-interface}

\begin{enumerate}
\item Log in to the AWS CLI.
\item Click your name in the top right and then Security Credentials.
\item Scroll down to the Access Keys section and hit Create access key.
\item For now check the box beside \textit{I understand creating a root access key is not a best practice, but I still want to create one} prompt and hit Create access key.
\item In PowerShell run:
\begin{verbatim}
aws configure
\end{verbatim}
It will prompt first for an access key. 
\item Copy the Access key from your web browser using the copy button to the left. Paste into Powershell (right click pastes). Press enter. You should see a prompt for the secret key.
\item In your web browser press Show beside Secret access key.
\item Click the copy button left of the secret access key. Paste into Powershell (right click pastes). Press enter. You should see a prompt for default region name.
\item Type \texttt{eu-west-1} and press enter.
\item At the next prompt for output format, just press enter again.
\end{enumerate}

\subsection{Check script}\label{check-script}

Run the \texttt{lab\_checks.ps1} powershell script from the main folder to confirm that your environment is set up correctly.
You can ignore the SSH key warnings, we will set this up next.

\subsection{Saving credentials to onedrive}

The college maps your OneDrive as Drive O:.
I have provided scripts to let you save your credentials from the local machine to your OneDrive.
\begin{verbatim}
# Save credentials to onedrive
./save_credentials_to_onedrive.ps1

# Load credentials from onedrive
./load_credentials_from_onedrive.ps1
\end{verbatim}


\section{SSH key}

Last year we used SSH keys given to us by the AWS Academy environment.
This year we will generate and use our own.

\textbf{If you already have an SSH key that you use already:} you should continue to use it.
Talk to me \textbf{before} attempting any of the instructions below as some steps are unnecessary and may erase your key.

\subsection{Generating keys}

We will generate an SSH keypair using the following command:
\begin{verbatim}
ssh-keygen -t rsa -b 4096
\end{verbatim}
You can hit enter to bypass the passphrase requirement.
The default save location is fine.

\subsection{Importing your SSH key}

The following command will import the public key into your AWS account with the name \texttt{MAIN\_KEY} (ignore the line breaks):
\begin{verbatim}
aws ec2 import-key-pair --key-name MAIN_KEY
 --public-key-material fileb://~/.ssh/id_rsa.pub 
\end{verbatim}

\subsection{Saving keys to onedrive}

As with your AWS credentials, I have provided scripts to let you save your keys from the local machine to your OneDrive.
\begin{verbatim}
# Save key to onedrive
./save_key_to_onedrive.ps1

# Load key from onedrive
./load_key_from_onedrive.ps1
\end{verbatim}



\end{document}

