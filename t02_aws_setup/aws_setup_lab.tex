\chapter{AWS setup}
\label{ch:aws-setup}

The labs in this course will use Amazon Web Services, or AWS.
To do the labs in this course, you will need your own AWS account.

Note that this is not a philosophical endorsement of Amazon.
There are other cloud providers too - IBM, Google, Microsoft Azure, others.
Most of the concepts encountered in AWS translate to the others.

\section{Charges}
\label{sec:charges}

Almost all AWS services are chargeable.
Many services have a time-limited free tier.
We will stay almost entirely within the free tier. 

You will need a credit / debit card to sign up for AWS.
If you don't have one you should be able to use a prepaid card or use an online card like Revolut.
\textit{I have not tested this option.}

\section{Sign up for an AWS account}
\label{sec:signup}

If you already have an AWS account you should skip this section.

Sign up for an AWS account by visiting the link:\\
\url{https://portal.aws.amazon.com/billing/signup}

There is a 5-step process to signing up.

\section{Log in to the console}

Make sure that you can login to the AWS console using the credentials from the \autoref{sec:signup}.
The AWS console can be accessed at:\\
\url{https://aws.amazon.com/console/}

Bookmark the AWS console link in your Browser. You will need it often.

\section{Billing alarm}

As stated in \autoref{sec:charges} we will aim to stay within the free tier where possible.
To avoid any unexpected charges, you will set up a Billing alarm on your account.
This uses a service named CloudWatch.
You will also familiarise yourself with using AWS documentation to help you work with AWS.
The lab instructions \textit{will NOT} provide step-by-step guides!

Visit the \href{https://docs.aws.amazon.com/AmazonCloudWatch/latest/monitoring/monitor_estimated_charges_with_cloudwatch.html}{documentation for CloudWatch Billing alarms}.

\subsection{Enabling billing alerts}

Follow the 4 steps under ``To enable the monitoring of estimated charges''.

You should also turn on the ``Receive Free Tier Usage Alerts'' tickbox.

\subsection{Creating a billing alarm}

Follow instructions 1 to 10 (of the 13) under ``To create a billing alarm using the CloudWatch console''.
Recommended alarm level is 10 US dollars.

When you have done the 10 steps, click ``Create New Topic''.
A default Topic name will be created.
Press Create Topic.

Fill in your e-mail address in the box.
Then press Next to go on to the next screen.

Set the name to \texttt{BILLING\_ALARM} and continue to the next step. 

Then press Create Alarm.
You should see a green bar with a message like \texttt{Successfully created alarm BILLING\_ALARM}.

Check your e-mail and confirm the subscription.

\subsection{Checking your billing alarm}

To confirm that your billing alarm has been set up, you will run a provided script.
\begin{enumerate}
\item Click the CloudShell icon \texttt{(\textrangle\_)}.
  This will open a command shell, which will take some time to initialise.

\item Wait until you see something like:
\begin{verbatim}
[cloudshell-user@ip-10-0-187-141 ~]$ 
\end{verbatim}

\item Click the Actions button and then Upload File.

\item Select the file \texttt{check\_billing\_alarm.py} and hit Upload.
  Wait for the upload to complete.

\item Then type:
\begin{verbatim}
python3 ./check_billing_alarm.py
\end{verbatim}

\item Confirm that your billing alarm has been set up from the script.
  If not, fix it and re-run the script to confirm. 

\end{enumerate}
  
% \section{Setup AWS command-line
% interface}\label{setup-aws-command-line-interface}

% The AWS Command Line Interface is a client program that runs on your
% local PC to allow you to manage AWS resources from the command-line
% (PowerShell, Bash). We will use the command-line extensively in this
% module.

% \subsection{PowerShell execution
% policy}\label{powershell-execution-policy}

% PowerShell by default will not allow scripts to run that were downloaded
% from online. The following command will change this behaviour:

% \begin{verbatim}
% Set-ExecutionPolicy -ExecutionPolicy RemoteSigned -Scope CurrentUser
% \end{verbatim}

% \subsection{AWS CLI installation}\label{aws-cli-installation}

% If you're on a lab computer OR if you already have the AWS CLI
% installed, then skip ahead to .

% Install the command-line tools from \url{https://aws.amazon.com/cli/}

% \subsection{Config file setup}\label{sec:config-file-setup}

% There is a script file \texttt{setup\_config\_file.ps1} that will setup
% your configuration file for you. Run it once.

% \subsection{Access key setup}\label{sec:access-key-setup}

% \textbf{Needs to be done EVERY time you log in on a student account!}

% Log in to AWS academy. Go to
% \url{https://awsacademy.instructure.com/courses/8294/modules/items/794040}.

% Hit the \emph{AWS Details} button. Look for \emph{Cloud Access} and
% \emph{AWS CLI} on the right. Click \emph{Show}. Copy this.

% \subsubsection{PowerShell}\label{powershell}

% You can paste the above using:

% \begin{verbatim}
% Get-Clipboard | Out-File ~/.aws/credentials
% \end{verbatim}

% Alternatively you can use the script:

% \begin{verbatim}
% .\paste_credentials.ps1
% \end{verbatim}

% \subsubsection{Manual alternative}\label{manual-alternative}

% Paste the contents into a file named EXACTLY

% \begin{verbatim}
% C:\Users\yourusername\.aws\credentials
% \end{verbatim}

% (no \texttt{.txt} etc at the end).

% \subsection{Check CLI configured}\label{check-cli-configured}

% To check that the AWS CLI is correctly configured, you can try running
% the command:

% \begin{verbatim}
% aws ec2 describe-instances
% \end{verbatim}

% If it shows something similar to:

% \begin{verbatim}
% {
%     "Reservations": []
% }
% \end{verbatim}

% then the AWS CLI is working OK.

% \section{Mac, Linux, UNIX users}\label{mac-linux-unix-users}

% \emph{Windows users can ignore this section.}

% Mac, Linux, Unix users will have no problems installing the AWS CLI, and
% the commands work identically to those on Windows.

% The difference between Mac/Linux and Windows centres on the use of
% Bash/zsh by Mac/Linux/UNIX vs PowerShell on Windows. The AWS CLI is
% perfectly scriptable using Bash, particularly in conjunction with
% \texttt{jq} to parse JSON. However, some of the scripts you will be
% provided with in this module will be PowerShell only due to time
% constraints.

% The good news is that PowerShell Core 7 can be installed easily on a Mac
% with no issues. You \emph{do not} need a Windows VM on your Mac to use
% any of the PowerShell or AWS commands in \emph{this} course. Please go
% to the \href{https://github.com/PowerShell/PowerShell}{PowerShell page
% on GitHub} for instructions.

% When you have PowerShell installed, open the Terminal app and type
% \texttt{pwsh} and you'll be at a PowerShell prompt. Repeat the
% \texttt{aws\ ec2\ describe-instances} command to confirm that the
% \texttt{aws} command is available in PowerShell.

% \section{Check script}\label{check-script}

% Run the \texttt{lab\_checks.ps1} powershell script to confirm that your
% environment is set up correctly.
