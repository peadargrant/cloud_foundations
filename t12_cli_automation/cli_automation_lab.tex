\documentclass{pgnotes}

\title{CLI automation lab}

\begin{document}

\maketitle

In this week's lab we are going to learn how to capture information from the AWS CLI output, rather than doing so by hand.

\section{JSON}

AWS CLI outputs JSON by default as you have seen so far.
Instead of manually copying and storing IDs, we can use our local PowerShell / Bash to capture this information for us.
This will help later on when we can write scripts to automate some of this work for us.

\subsection{JSON parsing in Powershell}

We will use the example of \texttt{create-vpc}.
\begin{minted}{powershell}
# Run aws ec2 create-vpc but capture the output instead
$Output = aws ec2 create-vpc --cidr-block 10.0.0.0/16

# View the output
$Output

# Convert from JSON to PS objects
$VpcInfo = $Output | ConvertFrom-Json

# Pull out the VpcId using the dot notation
$VpcId = $VpcInfo.Vpc.VpcId
\end{minted}

\subsubsection{Faster version}

\begin{minted}{powershell}
# Run aws ec2 create-vpc, capture output and convert to PS objects
$VpcInfo = aws ec2 create-vpc --cidr-block 10.0.0.0/16| ConvertFrom-Json 

# Pull out the VpcId using the dot notation
$VpcId = $VpcInfo.Vpc.VpcId
\end{minted}

\subsection{Even faster version}

If you know what the output of particular commands look like, and you're interested only in one parameter you can do the store immediately.
Consider instead:
\begin{minted}{powershell}
$VpcId = (aws ec2 create-vpc --cidr-block 10.0.0.0/16 | ConvertFrom-Json).Vpc.VpcId
\end{minted}

\subsection{JSON parsing in Bash}

Bash does not natively understand JSON but the \texttt{jq} utility does.

\section{Lab exercise}

Make sure you can login to the AWS Console.
Then use the CLI to do the following \textbf{without manually copying any IDs or other information!}:
\begin{enumerate}
\item Create a VPC \texttt{LAB\_VPC}, IP range 10.0.0.0/16. Store the ID as \texttt{\$VpcId}. 
\item Create a Subnet \texttt{LAB\_SUBNET\_1} within your VPC, IP range 10.0.1.0/24. Store the ID as \texttt{\$SubnetId}. 
\item Turn on auto IP address assignment in the subnet.
\item Create an Internet Gateway \texttt{LAB\_GATEWAY}. Store the ID as \texttt{\$GatewayId}. 
\item Attach the internet gateway to your VPC.
\item Get the route table ID and store it as \texttt{\$RouteTableId}. 
\item Alter the route table to send traffic for anywhere \texttt{0.0.0.0/0} to the internet gateway.
\item Create a security group \texttt{LAB\_GROUP} storing its ID as \texttt{\$GroupId}. 
\item Modify the security group to permit traffic inbound on SSH (22) and RDP (3389) protocols from anywhere \texttt{0.0.0.0/0}.
\item Create an instance \texttt{LAB\_INSTANCE} to run Linux. 
\item Wait until the instance is running.
\item Get the instance ID and store as \texttt{\$InstanceId}. 
\item Get the instance's public IP and store as \texttt{\$PublicIp}. 
\item Use SSH (linux) to login to the instance.
\end{enumerate}


\end{document}

